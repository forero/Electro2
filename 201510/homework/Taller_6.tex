\documentclass[letterpaper,10pt,onecolumn]{article}
\usepackage[spanish]{babel}
\usepackage[latin1]{inputenc}
\usepackage{amsfonts}
\usepackage{amsthm}
\usepackage{amsmath}
\usepackage{mathrsfs}
\usepackage{empheq}
\usepackage{enumitem}
\usepackage[pdftex]{color,graphicx}
\usepackage{hyperref}
\usepackage{listings}
\usepackage{calligra}
\usepackage{algpseudocode} 
\DeclareMathAlphabet{\mathcalligra}{T1}{calligra}{m}{n}
\DeclareFontShape{T1}{calligra}{m}{n}{<->s*[2.2]callig15}{}
\newcommand{\scripty}[1]{\ensuremath{\mathcalligra{#1}}}
\lstloadlanguages{[5.2]Mathematica}
\setlength{\oddsidemargin}{0cm}
\setlength{\textwidth}{490pt}
\setlength{\topmargin}{-40pt}
\addtolength{\hoffset}{-0.3cm}
\addtolength{\textheight}{4cm}

\begin{document}
\begin{center}

\includegraphics[width=490pt]{header.png}\\[0.5cm]

\textsc{\LARGE Taller 6 - Electromagnetismo II (FISI-3434) - 2015-10}\\[0.5cm]

\textsc{\Large{Profesor: Jaime Forero}} \\[0.5cm]

\textsc{Marzo 5, 2015} \\[0.5cm]

\end{center}

La soluci\'on a estos problemas va a ser evaluada (en el tablero) en
clase el martes 10 de marzo. 

\begin{enumerate}
\item Problema 9.16 de Griffiths. An\'alisis para
  una onda polarizada perpendicularmente al plano de incidencia.
\item Problema 9.17 de Griffiths. Propagaci\'on de luz aire/diamante.
\item Dos medios diel\'ectricos semi-infinitos est\'an separados por
  una franja de vaci\'o de ancho $d$. Calcule los coeficientes de
  transmisi\'on y reflecci\'on para una onda plana incidente desde
  unos de los medios con \'angulo de incidencia $\theta_I$, con
  polarizaci\'on paralela a las interfaces. 
\item Una onda plana es reflejada por un \'angulo $\theta$ medido
  desde la normal. Encuentre la presi\'on de radiaci\'on.

\end{enumerate}



\end{document}
