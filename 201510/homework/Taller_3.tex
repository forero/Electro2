\documentclass[letterpaper,10pt,onecolumn]{article}
\usepackage[spanish]{babel}
\usepackage[latin1]{inputenc}
\usepackage{amsfonts}
\usepackage{amsthm}
\usepackage{amsmath}
\usepackage{mathrsfs}
\usepackage{empheq}
\usepackage{enumitem}
\usepackage[pdftex]{color,graphicx}
\usepackage{hyperref}
\usepackage{listings}
\usepackage{calligra}
\usepackage{algpseudocode} 
\DeclareMathAlphabet{\mathcalligra}{T1}{calligra}{m}{n}
\DeclareFontShape{T1}{calligra}{m}{n}{<->s*[2.2]callig15}{}
\newcommand{\scripty}[1]{\ensuremath{\mathcalligra{#1}}}
\lstloadlanguages{[5.2]Mathematica}
\setlength{\oddsidemargin}{0cm}
\setlength{\textwidth}{490pt}
\setlength{\topmargin}{-40pt}
\addtolength{\hoffset}{-0.3cm}
\addtolength{\textheight}{4cm}

\begin{document}
\begin{center}

\includegraphics[width=490pt]{header.png}\\[0.5cm]

\textsc{\LARGE Taller 3 - Electromagnetismo II (FISI-3434) - 2015-10}\\[0.5cm]

\textsc{\Large{Profesor: Jaime Forero}} \\[0.5cm]

\textsc{Febrero 7, 2015} \\[0.5cm]

\end{center}

Los libros de referencia de donde tom\'e estos problemas son:

\begin{itemize}
\item \textit{Relatividad Especial (problemas selectos)} de Juan Manuel Tejeiro, editado por la Universidad Nacional de Colombia.

\item \textit{A guide to physics problems} de Sideney B. Cahn y Boris
  E. Nadgorny publicado por Kluwer Academic Publishers. 

\end{itemize}
La soluci\'on a estos problemas va a ser evaluada (en el tablero) en
clase el jueves 12 y martes 17 de febrero. 

\begin{enumerate}

\item Una part\'icula de masa en reposo $m_0$ tiene una energ\'ia
  total $E$ y un momento $\vec{p}$ medido por un observador inercial
  $\Sigma$. Un segundo sistema de referencia $\Sigma^{\prime}$ se
  mueve con velocidad  $v$ en la direcci\'on del eje $x$ positivo
  respecto a $\Sigma$. Elegiendo $t=t^{\prime}=0$ cuando los ejes de
  los dos sistemas coinciden, encuentre la energ\'ia total
  $E^{\prime}$, el momentum  $\vec{p}^{\prime}$ y la energ\'ia
  cin\'etica $K^{\prime}$ medidos en  el sistema $\Sigma^{\prime}$
  t\'erminos de las cantidades $E$, $\vec{p}$ y $K$ medidas por el
  observador $\Sigma$.    

\item En un experimento de colisi\'on de dos part\'iculas ($\alpha$ y
  $\alpha$)emergen las dos part\'iculas iniciales m\'as una nueva
  part\'icula $\delta$. 
\begin{displaymath}
\alpha + \alpha \rightarrow \alpha + \alpha + \delta.
\end{displaymath}
Para que la creaci\'on de la part\'icula $\delta$ sea posible es
necesario que exista una cantidad de energ\'ia disponible para dar
cuenta, al menos, de su energ\'ia en reposo. 

Hay dos formas de hacer esa reacci\'on. En la primera una 
part\'icula $\alpha$ tiene una energ\'ia propia $E_{0}$ y se acelera
hasta alcanzar una energ\'ia total $E_{1}$ para hacerla chocar contra
la otra part\'icula $\alpha$ que se encuentra en reposo. En la segunda
forma las dos part\'iculas se aceleran cada una hasta tener una
energ\'ia final $E_2$ y se hacen colisionar frontalmente. En ambos casos se
puede hacer que la energ\'ia total del sistema sea igual
$E=E_{1}=2E_{2}$, sin embargo las energ\'ias disponibles son
diferentes. Calcule la energ\'ia disponible $E_D$ para cada
experimento.

\item Demuestre que los siguientes procesos de colisi\'on son
  prohibidos:
\begin{itemize}
\item Un prot\'on en reposo emite un fot\'on y retrocede.
\item Un fot\'on de desintegra en un par electr\'on-positr\'on.
\item Un par electr\'on-positr\'on se aniquila dando lugar a un
  fot\'on. 
\end{itemize}

\item 
Un \'atomo en su estado de energ\'ia m\'as baja tiene una masa
$m$. Inicialmente en reposo es excitado a un estado con energ\'ia
$\Delta E$. Luego de esto hace una transici\'on a su estado base
emitiendo un fot\'on de esta misma energ\'ia $\Delta E$. Encuentre la
frecuencia del fot\'on tomando en cuenta el retroceso relativista del
\'atomo por conservaci\'on del 4-momento. Exprese su respuesta en
t\'erminos de la masa $M$ del \'atomo excitado.

\item Un electr\'on $e^{-}$ y un positr\'on $e^+$, cada uno de masa en
  reposo $m_e$ se encuentran unidos con una energ\'ia de ligadura
  $E_b$ formando positronio. Estando en esta configuraci\'on se
  aniquilan formando dos fotones. Calcule la energ\'ia, momentum y
  frecuencia de los fotones. 

\item Ahora el positronio del caso anterior se mueve a una velocidad
  $\vec{v}$ alej\'andose de un observador en un sistema $\Sigma$ antes
  de aniquilarse. Encuentre la frecuencia $\omega$ de los dos fotones
  medida en el sistema $\Sigma$ en t\'erminos de la frecuencia
  $\omega_0$ medida en el sistema en  reposo del positronio. 

\item Una diferencia entre mec\'anica cl\'asica y relatividad  es la
  existencia de un efecto Doppler transversal en relatividad cuando la
  luz se propaga de manera perpendicular a la fuente en el marco de
  referencia del observador. Calcule para este caso la frecuencia del fot\'on
  $\omega^\prime$  en t\'ermino de la frecuencia $\omega$ en su
  sistema de reposo. 


\item Un fot\'on de energ\'ia $Q_i$ choca contra un \'atomo de
  hidr\'ogeno en reposo que se enceuntra en su primer estado
  excitado. Despu\'es de la colisi\'on el \'atomo pasa a su estado
  base y el fot\'on se dispersa con la misma ener\'ia retrocediento
  (i.e. se refleja contra el \'atomo).
  \begin{itemize}
    \item �Este tipo de colisi\'on es el\'astica o inel\'astica?
    \item �Cu\'al es la velocidad del \'atomo despu\'es de la colisi\'on?
    \item �Cu\'al es la energ\'ia $Q_i$ del fot\'on incidente?
\end{itemize}


\end{enumerate}


\end{document}
