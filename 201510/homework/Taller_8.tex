\documentclass[letterpaper,10pt,onecolumn]{article}
\usepackage[spanish]{babel}
\usepackage[latin1]{inputenc}
\usepackage{amsfonts}
\usepackage{amsthm}
\usepackage{amsmath}
\usepackage{mathrsfs}
\usepackage{empheq}
\usepackage{enumitem}
\usepackage[pdftex]{color,graphicx}
\usepackage{hyperref}
\usepackage{listings}
\usepackage{calligra}
\usepackage{algpseudocode} 
\DeclareMathAlphabet{\mathcalligra}{T1}{calligra}{m}{n}
\DeclareFontShape{T1}{calligra}{m}{n}{<->s*[2.2]callig15}{}
\newcommand{\scripty}[1]{\ensuremath{\mathcalligra{#1}}}
\lstloadlanguages{[5.2]Mathematica}
\setlength{\oddsidemargin}{0cm}
\setlength{\textwidth}{490pt}
\setlength{\topmargin}{-40pt}
\addtolength{\hoffset}{-0.3cm}
\addtolength{\textheight}{4cm}

\begin{document}
\begin{center}

\includegraphics[width=490pt]{header.png}\\[0.5cm]

\textsc{\LARGE Taller 8 - Electromagnetismo II (FISI-3434) - 2015-10}\\[0.5cm]

\textsc{\Large{Profesor: Jaime Forero}} \\[0.5cm]

\textsc{Marzo 28, 2015} \\[0.5cm]

\end{center}

La soluci\'on a estos problemas va a ser evaluada (en el tablero) en
clase el jueves 16 de abril.

\begin{enumerate}
\item (Quiz \#6) Escriba un programa que haga gr\'aficas de los lugares donde el campo el\'ectrico tiene valores constantes para una carga el\'ectrica $q/4\pi\epsilon_0=1N m^2$ que se mueve a con $\beta$ arbitratio. $\beta$ es un valor que el usuario debe poder cambiar. Fecha l\'imite de entrega por SICUA: Jueves 16 de abril a las 9AM.

\item Problema 10.18 Griffiths. 
\item Encuentre los potenciales para un dipolo el\'ectrico en reposo
  que var\'ia en el tiempo. 
\item Encuentre los campos para un dipolo el\'ectrico en reposo que
  var\'ia en el tiempo.  
\end{enumerate}



\end{document}
