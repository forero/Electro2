\documentclass[letterpaper,10pt,onecolumn]{article}
\usepackage[spanish]{babel}
\usepackage[latin1]{inputenc}
\usepackage{amsfonts}
\usepackage{amsthm}
\usepackage{amsmath}
\usepackage{mathrsfs}
\usepackage{empheq}
\usepackage{enumitem}
\usepackage[pdftex]{color,graphicx}
\usepackage{hyperref}
\usepackage{listings}
\usepackage{calligra}
\usepackage{algpseudocode} 
\DeclareMathAlphabet{\mathcalligra}{T1}{calligra}{m}{n}
\DeclareFontShape{T1}{calligra}{m}{n}{<->s*[2.2]callig15}{}
\newcommand{\scripty}[1]{\ensuremath{\mathcalligra{#1}}}
\lstloadlanguages{[5.2]Mathematica}
\setlength{\oddsidemargin}{0cm}
\setlength{\textwidth}{490pt}
\setlength{\topmargin}{-40pt}
\addtolength{\hoffset}{-0.3cm}
\addtolength{\textheight}{4cm}

\begin{document}
\begin{center}

\includegraphics[width=490pt]{header.png}\\[0.5cm]

\textsc{\LARGE Taller 11 - Electromagnetismo II (FISI-3434) - 2015-10}\\[0.5cm]

\textsc{\Large{Profesor: Jaime Forero}} \\[0.5cm]

\textsc{Abril 30 2015} \\[0.5cm]

\end{center}

La soluci\'on a estos problemas va a ser evaluada (en el tablero) en
clase el jueves 7 de mayo.

\begin{enumerate}
\item (Quiz \#8) Escriba un programa que haga gr\'aficas (2D)de la
  distribuci\'on angular de radiaci\'on para una part\'icula
  relativista con acelaraci\'on constante en dos casos: aceleraci\'on
  perpendicular a la velocidad y aceleraci\'on paralela a la velocidad.
  Fecha l\'imite de entrega por SICUA: Jueves 7 de mayo a las 9AM. 

\item Griffiths 11.9

\item Griffiths 11.10

\item Griffiths 11.15

\item Griffiths 11.26

\item Jackson 9.16

\item Jackson 14.3

\item Jackson 14.7 a)

\item Jackson 14.8

\item Jackson 14.9

\end{enumerate}



\end{document}
