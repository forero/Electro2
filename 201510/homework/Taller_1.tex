\documentclass[letterpaper,10pt,onecolumn]{article}
\usepackage[spanish]{babel}
\usepackage[latin1]{inputenc}
\usepackage{amsfonts}
\usepackage{amsthm}
\usepackage{amsmath}
\usepackage{mathrsfs}
\usepackage{empheq}
\usepackage{enumitem}
\usepackage[pdftex]{color,graphicx}
\usepackage{hyperref}
\usepackage{listings}
\usepackage{calligra}
\usepackage{algpseudocode} 
\DeclareMathAlphabet{\mathcalligra}{T1}{calligra}{m}{n}
\DeclareFontShape{T1}{calligra}{m}{n}{<->s*[2.2]callig15}{}
\newcommand{\scripty}[1]{\ensuremath{\mathcalligra{#1}}}
\lstloadlanguages{[5.2]Mathematica}
\setlength{\oddsidemargin}{0cm}
\setlength{\textwidth}{490pt}
\setlength{\topmargin}{-40pt}
\addtolength{\hoffset}{-0.3cm}
\addtolength{\textheight}{4cm}

\begin{document}
\begin{center}

\includegraphics[width=490pt]{header.png}\\[0.5cm]

\textsc{\LARGE Taller 1 - Electromagnetismo II (FISI-3434) - 2015-10}\\[0.5cm]

\textsc{\Large{Profesor: Jaime Forero}} \\[0.5cm]

\end{center}

Los libros de referencia de donde tom\'e estos problemas son:

\begin{itemize}
\item \textit{Relatividad para estudiantes de f\'isica} de Shahen Hacyan editado por el Fondo de Cultura Econ\'omica.

\item \textit{Relatividad Especial (problemas selectos)} de Juan Manuel Tejeiro, editado por la Universidad Nacional de Colombia.

\item \textit{Fundamentals of Electro-magnetism. Vacuum Electrodynamics, Media and Relativity} de Antonio Lopez D\'avalos y Dami\'an Zanette editado por Springer.

La soluci\'on a estos problemas va a ser evaluada (en el tablero) en clase la semana que viene. 

\end{itemize}

\begin{enumerate}
\item Visto desde un sistema inercial $\Sigma$, una part\'icula se mueve en un c\'irculo de radio $R$ con velocidad angular $\omega$ con $R$ y $\omega$ medidos en $\Sigma$. En t\'erminos de $R$, $\omega$ y $c$ �en qu\'e factor se reduce el tiempo propotio de la part\'icula con respecto al tiempo medido en $\Sigma$? (Hacyan).

\item El mu\'on es una part\'icula elemental que se puede producir al llegar un rayo c\'osmico a los estratos m\'as altos de la atm\'osfera terrestre. La vida media de los muones es de aproximadamente $2\times 10^{-6}$ segundos, y se han detectado con velocidades del orden del $0.997$ veces la velocidad de la luz. Aun a esa velocidad, su vida media no le permitir\'ia recorre m\'as que unos $600$ metros, seg\'un la f\'isica cl\'asica. Demuestre que, de acuerdo con la teor\'ia de la relatividad, la distancia que recorre es mucho mayor y suficiente para llegar a la superficie de la Tierra. (Hacyan). 

\item Olga y Luc\'ia son dos gemelas.  Un d\'ia, Luc\'ia aborda una nave espacial que la lleva a una velocidad $V$ (de magnitud constante) cercana a la de la luz a una estrella que se encuntra a $L$ a\~nos luz. Demuestre que, a su regreso, Luc\'ia es m\'as joven que su hermana Olga, quien se qued\'o en la Tierra. �Cu\'al es la diferencia de las edades en funci\'on de $V$ y $L$? Desprecie los tiempos de aceleraci\'on y desaceleraci\'on de la nave. (Hacyan)

\item Consideremos un bus de longitud propia $l_0=10$m que se mueve a una velocidad de $v=0.8c$, directamente hacia un garaje en reposo de longitud de $6$m. DEbido al efecto de la contracci\'on de longitudes el bus mide, respecto al sistema de referencia del garaje, 
\begin{displaymath}
l=l_0\sqrt{1-\frac{v^2}{c^2}} =6\mathrm{m}.
\end{displaymath}

As\'i, cuando la parte de adelante del bus alcanza la pared del fondo del garaje, la parte de atr\'as del bus pasa por la puerta. Respecto a un observador fijo con relaci\'on al garaje, el bus cabe dentro del garaje. Analizando la misma situaci\'on desde el punto de vista donde el bus est\'a en reposo, el garaje se mueve hacia el bus con velocidad $0.8c$ y la longitud del garaje es de $3.6$m y el bus ya no cabr\'ia dentro del garaje. �Cabe o no el carro dentro del garaje? �C\'omo se resuelve esta aparante paradoja? (Tejeiro)

\item Consideremos una varilla de longitud propia $l_0$ situada en el plano $x-y$ que se mueve con velocidad $v$ a lo largo del eje $x$ positivo respecto a un observador inercial. La varilla forma un \'angulo $\theta_0$ en su sistema de referencia propio con respecto al eje de las $x$. �Que \'angulo forma con respecto al observador inercial $\Sigma$? (Tejeiro)

\item Una fuente isotr\'opica en reposo emite part\'iculas de velocidad $v$ de manera pulsada con un per\'iodo $T$. Cada pulso determina un frente esf\'erico de part\'iculas con un radio que crece con el tiempo. Encuentre la forma del frente de part\'iculas y la distancia entre frentes sucesivos que medir\'ia un observador que se mueve a velocidad $V$ con respecto a la fuente de part\'iculas. (L\'opez D\'avalos \& Zanette)

\end{enumerate}



\end{document}
