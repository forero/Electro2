\documentclass[letterpaper,10pt,onecolumn]{article}
\usepackage[spanish]{babel}
\usepackage[latin1]{inputenc}
\usepackage{amsfonts}
\usepackage{amsthm}
\usepackage{amsmath}
\usepackage{mathrsfs}
\usepackage{empheq}
\usepackage{enumitem}
\usepackage[pdftex]{color,graphicx}
\usepackage{hyperref}
\usepackage{listings}
\usepackage{calligra}
\usepackage{algpseudocode} 
\DeclareMathAlphabet{\mathcalligra}{T1}{calligra}{m}{n}
\DeclareFontShape{T1}{calligra}{m}{n}{<->s*[2.2]callig15}{}
\newcommand{\scripty}[1]{\ensuremath{\mathcalligra{#1}}}
\lstloadlanguages{[5.2]Mathematica}
\setlength{\oddsidemargin}{0cm}
\setlength{\textwidth}{490pt}
\setlength{\topmargin}{-40pt}
\addtolength{\hoffset}{-0.3cm}
\addtolength{\textheight}{4cm}

\begin{document}
\begin{center}

\includegraphics[width=490pt]{header.png}\\[0.5cm]

\textsc{\LARGE Taller 10 - Electromagnetismo II (FISI-3434) - 2015-10}\\[0.5cm]

\textsc{\Large{Profesor: Jaime Forero}} \\[0.5cm]

\textsc{Abril 916 2015} \\[0.5cm]

\end{center}

La soluci\'on a estos problemas va a ser evaluada (en el tablero) en
clase el jueves 23 de abril.

\begin{enumerate}
\item (Quiz \#7) Escriba un programa que haga gr\'aficas de los lugares
  donde el campo el\'ectrico tiene valores constantes para dos cargas 
  el\'ectricas $\pm q/4\pi\epsilon_0=1N m^2$ ubicadas en forma de
  dipolo separadas una distancia $3\times 10^{-2}$m y que oscilan con
  una frecuencia angular de $\omega = 2\pi$ rad/s (usando la
  convenci\'on de Griffiths para las cargas en funci\'on del
  tiempo). Estas gr\'aficas se deben hacer 
  para los siguientes tiempos $t=0, 1/4\times 2\pi\omega^{-1},
  1/2\times 2\pi\omega^{-1}$.  Fecha l\'imite de entrega por SICUA:
  Jueves 23 de abril a las 9AM. 

\item Encuentre la ecuaci\'on de movimiento para el campo escalar
  $\phi$ que tiene las siguientes densidades lagrangianas
\begin{itemize}
\item ${\mathcal L} = \frac{1}{2}\phi\phi - \frac{1}{2}|\nabla\phi|^2$
\item ${\mathcal L} = \frac{1}{2}(\partial_\mu\phi)(\partial_\mu\phi) -
  \frac{1}{2}\sigma\phi^2 $
\end{itemize}


\item Encuentre las ecuaciones de movimiento correspondientes a la
  siguiente densidad Lagrangiana.
\begin{equation}
{\mathcal L} = -\frac{1}{8\pi}\partial_\alpha A_\beta
\partial^{\alpha}A^{\beta} -\frac{1}{c}J_{\alpha}A^{\alpha}.
\end{equation}
C\'omo se relacionan esas ecuaciones de movimiento con las ecuaciones
de Maxwell?

\item Encuentre las ecuaciones de movimiento correspondientes a la
  siguiente densidad Lagrangiana.
\begin{equation}
{\mathcal L} = -\frac{1}{4}F_{\mu\nu}F^{\mu\nu} -
\frac{a^2}{2}\partial_\mu F^{\alpha\mu}\partial^{\beta}F_{\alpha\beta}
\end{equation}
C\'omo se relacionan esas ecuaciones de movimiento con las ecuaciones
de Maxwell?

\item Muestre expl\'icitamente que el Lagrangiano $L({\rm r, {\rm
    v}})=-mc^2/\gamma - q\Phi + q{\bf v}\cdot{\bf A}$ predice la
  ecuaci\'on relativista correcta para el movimiento de una
  part\'icula con carga $q$ y masa $m$.

\item Muestre expl\'icitamente que el tensor electromagn\'etico
  $F^{\mu\nu}$ es efectivamente un tensor dos veces contravariante.

\end{enumerate}



\end{document}
