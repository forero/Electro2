\documentclass[letterpaper,10pt,onecolumn]{article}
\usepackage[spanish]{babel}
\usepackage[latin1]{inputenc}
\usepackage{amsfonts}
\usepackage{amsthm}
\usepackage{amsmath}
\usepackage{mathrsfs}
\usepackage{empheq}
\usepackage{enumitem}
\usepackage[pdftex]{color,graphicx}
\usepackage{hyperref}
\usepackage{listings}
\usepackage{calligra}
\usepackage{algpseudocode} 
\DeclareMathAlphabet{\mathcalligra}{T1}{calligra}{m}{n}
\DeclareFontShape{T1}{calligra}{m}{n}{<->s*[2.2]callig15}{}
\newcommand{\scripty}[1]{\ensuremath{\mathcalligra{#1}}}
\lstloadlanguages{[5.2]Mathematica}
\setlength{\oddsidemargin}{0cm}
\setlength{\textwidth}{490pt}
\setlength{\topmargin}{-40pt}
\addtolength{\hoffset}{-0.3cm}
\addtolength{\textheight}{4cm}

\begin{document}
\begin{center}

\includegraphics[width=490pt]{header.png}\\[0.5cm]

\textsc{\LARGE Taller 2 - Electromagnetismo II (FISI-3434) - 2015-10}\\[0.5cm]

\textsc{\Large{Profesor: Jaime Forero}} \\[0.5cm]

\textsc{Enero 29, 2015} \\[0.5cm]

\end{center}

Los libros de referencia de donde tom\'e estos problemas son:

\begin{itemize}
\item \textit{Relatividad Especial (problemas selectos)} de Juan Manuel Tejeiro, editado por la Universidad Nacional de Colombia.

\item \textit{Fundamentals of Electro-magnetism. Vacuum Electrodynamics, Media and Relativity} de Antonio Lopez D\'avalos y Dami\'an Zanette editado por Springer.

La soluci\'on a estos problemas va a ser evaluada (en el tablero) en clase la semana que viene. 

\end{itemize}

\begin{enumerate}

\item  Considere dos part\'iculas que se mueven con velocidades
  $\vec{v}_1$ y $\vec{v}_2$ respecto a un observador inercial
  $\Sigma$. Calcular la magnitud de la velocidad relativa entre las
  dos part\'iculas en t\'erminos de $\vec{v}_1$ y $\vec{v}_2$.  

\item Encontrar la relaci\'on entre las componentes del cuadrivector aceleraci\'on
\begin{displaymath}
A =(A^0, A^1, A^2, A^3)
\end{displaymath}
y las componentes de la acelaraci\'on f\'isica de una part\'icula $\vec{a}=\frac{d\vec{u}}{dt}$.


\item Una nave espacial sale de la Tierra con acelaraci\'on propia
  de magnitud constante $a=10$m s$^{-2}$
  y se dirige hacia una estrella ubicada a $10$ a\~nos luz.
\begin{itemize}
\item Calcular el cuadrivector posici\'on para el cohete en funci\'on
  del tiempo propio de la nave en el sistema de referencia de la
  Tierra, asumiendo que cuando la nave sale $t=0$ en el sistema de
  referencia de la Tierra y $\tau=0$ en el tiempo propido de la nave. 
\item Supongamos que el cohete acelera hasta llegar a la mitad de
  camino hasta la estrella. �Qu� velocidad tiene en ese momento con
  respecto a la Tierra?
\item Supongamos que el cohete empieza a desacelerar en ese momento
  hasta llegar a la estrella. Si la nave regresa a la Tierra siguiendo
  un recorrido similar al de ida. �Cu�nto tiempo tarda en total el
  viaje para un observador en la Tierra?
\end{itemize}

\item Tres fuentes de luz en reposo en un sistema de referencia
  tridimensional $\Sigma$ se encienden simultaneamente. Otro sistema
  inercial de referencia $\Sigma^\prime$ se mueve con respecto a
  $\Sigma$. 
\begin{itemize}
  \item Encuentre los puntos $P^\prime$ en $\Sigma^\prime$ desde
  los cuales las fuentes de luz parece que se encendieran
  simult\'aneamente. 
  \item Calcule las distancias medidas en $\Sigma^\prime$ entre
    $P^{\prime}$ hasta las fuentes de luz y los tiempos de llegada de
    la se�al hasta $P^{\prime}$. �C\'omo se comparan con las
    distancias y tiempos correspondientes medidos en $\Sigma$?
  \item �Qu\'e pasar\'ia si hubiera $N$ fuentes de luz?
  \item �Qu\'e pasar\'ia si el espacio tuviera $D$
    dimensiones? 
\end{itemize}

\item Dos trenes comparten un mismo riel y est\'an separados una
  distancia $D$. Los trenes, vistos desde la plataforma de la
  estaci\'on, van en colisi\'on frontal, cada uno movi\'endose con una
  rapidez $v$.  
  \begin{itemize}
    \item Calcule el tiempo que le toma a los trenes chocar, tanto en
      el sistema de referencia de la plataforma como para un sistema
      de referencia que se mueve en uno de los trenes. �Se satisface
      la regla de dilataci\'on temporal en este caso? �Por qu\'e?
    \item Una mosca viaja a una velocidad $V>v$ entre las cabezas de
      los trenes, toc\'andolos sucesivamente hasta que es aplastada
      por la colisi\'on. Calcule la distancia que viaja la mosca antes
      de ser aplastada tanto para el sistema de referencia de la
      plataforma como otro que se mueve con uno de los trenes. �Se
      satisface la contracci\'on de longitudes? �Por qu\'e?
  \end{itemize}


\end{enumerate}



\end{document}
