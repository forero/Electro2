\documentclass[letterpaper,10pt,onecolumn]{article}
\usepackage[spanish]{babel}
\usepackage[latin1]{inputenc}
\usepackage{amsfonts}
\usepackage{amsthm}
\usepackage{amsmath}
\usepackage{mathrsfs}
\usepackage{empheq}
\usepackage{enumitem}
\usepackage[pdftex]{color,graphicx}
\usepackage{hyperref}
\usepackage{listings}
\usepackage{calligra}
\usepackage{algpseudocode} 
\DeclareMathAlphabet{\mathcalligra}{T1}{calligra}{m}{n}
\DeclareFontShape{T1}{calligra}{m}{n}{<->s*[2.2]callig15}{}
\newcommand{\scripty}[1]{\ensuremath{\mathcalligra{#1}}}
\lstloadlanguages{[5.2]Mathematica}
\setlength{\oddsidemargin}{0cm}
\setlength{\textwidth}{490pt}
\setlength{\topmargin}{-40pt}
\addtolength{\hoffset}{-0.3cm}
\addtolength{\textheight}{4cm}

\begin{document}
\begin{center}

\includegraphics[width=490pt]{header.png}\\[0.5cm]

\textsc{\LARGE Taller 2 - Electromagnetismo II (FISI-3434) - 2015-10}\\[0.5cm]

\textsc{\Large{Profesor: Jaime Forero}} \\[0.5cm]

\textsc{Enero 26, 2015} \\[0.5cm]

\end{center}

Los libros de referencia de donde tom\'e estos problemas son:

\begin{itemize}
\item \textit{Relatividad para estudiantes de f\'isica} de Shahen Hacyan editado por el Fondo de Cultura Econ\'omica.

\item \textit{Relatividad Especial (problemas selectos)} de Juan Manuel Tejeiro, editado por la Universidad Nacional de Colombia.

\item \textit{Fundamentals of Electro-magnetism. Vacuum Electrodynamics, Media and Relativity} de Antonio Lopez D\'avalos y Dami\'an Zanette editado por Springer.

La soluci\'on a estos problemas va a ser evaluada (en el tablero) en clase la semana que viene. 

\end{itemize}

\begin{enumerate}
\item A.

\end{enumerate}



\end{document}
