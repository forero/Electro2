\documentclass[letterpaper,10pt,onecolumn]{article}
\usepackage[spanish]{babel}
\usepackage[latin1]{inputenc}
\usepackage{amsfonts}
\usepackage{amsthm}
\usepackage{amsmath}
\usepackage{mathrsfs}
\usepackage{empheq}
\usepackage{enumitem}
\usepackage[pdftex]{color,graphicx}
\usepackage{hyperref}
\usepackage{listings}
\usepackage{calligra}
\usepackage{algpseudocode} 
\DeclareMathAlphabet{\mathcalligra}{T1}{calligra}{m}{n}
\DeclareFontShape{T1}{calligra}{m}{n}{<->s*[2.2]callig15}{}
\newcommand{\scripty}[1]{\ensuremath{\mathcalligra{#1}}}
\lstloadlanguages{[5.2]Mathematica}
\setlength{\oddsidemargin}{0cm}
\setlength{\textwidth}{490pt}
\setlength{\topmargin}{-40pt}
\addtolength{\hoffset}{-0.3cm}
\addtolength{\textheight}{4cm}

\begin{document}
\begin{center}

\includegraphics[width=490pt]{header.png}\\[0.5cm]

\textsc{\LARGE Taller 7 - Electromagnetismo II (FISI-3434) - 2015-10}\\[0.5cm]

\textsc{\Large{Profesor: Jaime Forero}} \\[0.5cm]

\textsc{Marzo 19, 2015} \\[0.5cm]

\end{center}

La soluci\'on a estos problemas va a ser evaluada (en el tablero) en
clase el jueves 26 de marzo. 

\begin{enumerate}
\item Problema 10.10 de Griffiths.
\item Encuentre en los potenciales (vectoriales y escalar) de una
  part\'icula cargada que se mueve a velocidad constante en la
  direcci\'on $z$ de tres maneras diferentes.
\begin{itemize}
\item Resolviendo directamente la ecuaci\'on de onda correspondiente.
\item Integrando las distribuciones $\rho$ y $\vec{{\bf j}}$ descritas
  por deltas de Dirac. 
\item Integrando las distribuciones $\rho$ y $\vec{{\bf j}}$ donde las
  distribuciones corresponden a un cubito de carga de lado $l$ y luego
  tome el l\'imite $l\rightarrow 0$. 
\end{itemize}
\item Encuentre los campos el\'ectricos y magn\'eticos para la misma
  part\'icula del punto anterior.
\end{enumerate}



\end{document}
