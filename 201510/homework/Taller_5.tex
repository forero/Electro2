\documentclass[letterpaper,10pt,onecolumn]{article}
\usepackage[spanish]{babel}
\usepackage[latin1]{inputenc}
\usepackage{amsfonts}
\usepackage{amsthm}
\usepackage{amsmath}
\usepackage{mathrsfs}
\usepackage{empheq}
\usepackage{enumitem}
\usepackage[pdftex]{color,graphicx}
\usepackage{hyperref}
\usepackage{listings}
\usepackage{calligra}
\usepackage{algpseudocode} 
\DeclareMathAlphabet{\mathcalligra}{T1}{calligra}{m}{n}
\DeclareFontShape{T1}{calligra}{m}{n}{<->s*[2.2]callig15}{}
\newcommand{\scripty}[1]{\ensuremath{\mathcalligra{#1}}}
\lstloadlanguages{[5.2]Mathematica}
\setlength{\oddsidemargin}{0cm}
\setlength{\textwidth}{490pt}
\setlength{\topmargin}{-40pt}
\addtolength{\hoffset}{-0.3cm}
\addtolength{\textheight}{4cm}

\begin{document}
\begin{center}

\includegraphics[width=490pt]{header.png}\\[0.5cm]

\textsc{\LARGE Taller 5 - Electromagnetismo II (FISI-3434) - 2015-10}\\[0.5cm]

\textsc{\Large{Profesor: Jaime Forero}} \\[0.5cm]

\textsc{Febrero 26, 2015} \\[0.5cm]

\end{center}

%Los problemas salieron del siguiente libro:
%\begin{itemize}
%\item \textit{Modern Electrodynamics} de Andrew Zangwill, editado por
%  Cambridge University Press. 
%\end{itemize}

La soluci\'on a estos problemas va a ser evaluada (en el tablero) en
clase el martes 3 de marzo. El c\'odigo del punto 4 va a contar como
un quiz que se debe entregar por SICUA antes del 7 de marzo a las 7AM.

\begin{enumerate}
\item

Encuentre y describa el campo electromagn\'etico producido por la
superposici\'on de dos ondas planas de igual amplitud,
monocrom\'aticas, que se propagan en direcciones opuestas. 
\begin{itemize}
\item
  Haga el
  ejercicio cuando la onda que se propaga en $+z$ tiene polarizaci\'on
  circular izquierda (i.e. la punta del vector ${\bf E}$ gira contrario
  a las manecillas del reloj) y cuando la onda que se propaga en $-z$
  tiene polarizacion circular derecha. 

\item Repita el ejercicio cuando ambas ondas tienen polarizaci\'on
  circular izquierda.
\end{itemize}


\item El campo el\'ectrico de una onda que se propaga en el vac\'io es
  ${\bf E}={\bf\hat{y}}E_0\exp[i(hz-\omega t)-\kappa x]$. 

\begin{itemize}
\item C\'omo est\'an relacionados entre s\'i los par\'ametros reales
  $h$, $\kappa$ y $\omega$?
\item Encuentre el campo magn\'etico asociado.
\item Calcule el vector de Poynting promediado en el tiempo.
\end{itemize}

\item 
Usando la siguiente forma de escribir el campo el\'ectrico de un
paquete de onda

\begin{displaymath}
{\bf E}({\bf r}, t)=\frac{1}{(2\pi)^{3/2}}\int d^3k {\bf
  E}_{\perp}\exp[i({\bf k}\cdot{\bf r} -ckt)]
\end{displaymath}
\begin{itemize}
\item
Muestre que el momento lineal total del paquete de onda satisface:
\begin{displaymath}
c{\bf P}_{EM}=\frac{1}{2}\epsilon_0\int {\bf \hat{k}}|{\bf
  E}_{\perp}|^2.
\end{displaymath}
\item
Justifique la siguiente desigualdad: $U_{EM}\geq c|{\bf P}_{EM}|$
\item
�Cu\'ando aplica la igualdad en el caso anterior? (Ese fue el caso que
  tratamos en clase).
\end{itemize}

\item Considere un \'atomo de Hidr\'ogeno ionizado en la parte alta de
  la atm\'osfera terrestre. Considerando que esa parte de la
  atm\'osfera la intensidad de la luz solar es aproximadamente $1300$
  W/m$^2$ y que la luz que llega est\'a
  linealmente polarizada escriba un programa (en C, Python o Java) que
  describa la trayectoria de este \'atomo de Hidr\'ogeno ionizado
  cuando interact\'ua con la luz solar. Haga expl\'icitas todas las
  aproximaciones que utilice para resolver el problema.
\end{enumerate}



\end{document}
