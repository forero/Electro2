\documentclass[letterpaper,10pt,onecolumn]{article}
\usepackage[spanish]{babel}
\usepackage[latin1]{inputenc}
\usepackage{amsfonts}
\usepackage{amsthm}
\usepackage{amsmath}
\usepackage{mathrsfs}
\usepackage{empheq}
\usepackage{enumitem}
\usepackage[pdftex]{color,graphicx}
\usepackage{hyperref}
\usepackage{listings}
\usepackage{calligra}
\usepackage{algpseudocode} 
\DeclareMathAlphabet{\mathcalligra}{T1}{calligra}{m}{n}
\DeclareFontShape{T1}{calligra}{m}{n}{<->s*[2.2]callig15}{}
\newcommand{\scripty}[1]{\ensuremath{\mathcalligra{#1}}}
\lstloadlanguages{[5.2]Mathematica}
\setlength{\oddsidemargin}{0cm}
\setlength{\textwidth}{490pt}
\setlength{\topmargin}{-40pt}
\addtolength{\hoffset}{-0.3cm}
\addtolength{\textheight}{4cm}

\begin{document}
\begin{center}

\includegraphics[width=490pt]{header.png}\\[0.5cm]

\textsc{\LARGE Taller 4 - Electromagnetismo II (FISI-3434) - 2015-10}\\[0.5cm]

\textsc{\Large{Profesor: Jaime Forero}} \\[0.5cm]

\textsc{Febrero 19, 2015} \\[0.5cm]

\end{center}

%Los problemas salieron del siguiente libro:
%\begin{itemize}
%\item \textit{Modern Electrodynamics} de Andrew Zangwill, editado por
%  Cambridge University Press. 
%\end{itemize}

La soluci\'on a estos problemas va a ser evaluada (en el tablero) en
clase el martes 3 de marzo.

\begin{enumerate}
\item Una antigua teor\'ia competidora del Big Bang postulaba la
  \emph{creaci\'on continua} de materia cargada a una (muy peque\~na) taza
  constante $R$ en  todos los puntos del espacio. En esta teor\'ia la
  ecuaci\'on de continuidad se deber\'ia reemplazar por

\begin{displaymath}
\nabla\cdot{\bf j} + \frac{\partial\rho}{\partial t} = R.
\end{displaymath}

Para que esto sea cierto es necesario modificar tambi\'en los
t\'erminos fuente en las ecuaciones de Maxwell. Muestre que es
suficiente moficar la ley de Gauss a

\begin{displaymath}
\nabla\cdot{\bf E} = \frac{\rho}{\varepsilon_0} - \lambda\varphi
\end{displaymath}

y la ley de Amp\`ere-Maxwell a

\begin{displaymath}
\nabla\times{\bf B} = \mu_0 {\bf j} + \frac{1}{c^2}\frac{\partial {\bf
E}}{\partial t} - \lambda{\bf A}.
\end{displaymath}
Donde $\lambda$ es una constante y $\varphi$ y ${\bf A}$ son los
vectores escalares y potenciales.


\item Una barra resistiva con masa $m$ se mueve sin fricci\'on sobre
  dos rieles conductores. Un campo magn\'etico ${\bf B}$ apunta
  saliendo de la p\'agina como muestra la figura. Sea $R$ la
  resistencia de la barra de longitud $l$. Pruebe que la energ\'ia
  cin\'etica inicial de la barra se disipa completamente como calor en
  el circuito cuando $t\rightarrow \infty$.
\begin{center}
\includegraphics[width=200pt]{barra.png}
\end{center}

\item Una carga puntual $q$ est\'a fija en las coordenadas  $(a,0,0)$, otra
  carga puntual $-q$ est\'a fija en $(-a,0,0)$. Adicionalmente un
  campo magn\'etico uniforme ${\bf B}=B{\bf \hat{z}}$ llena todo el
  espacio.
\begin{itemize}
\item  Pruebe que las lineas de flujo del vector de Poynting ya sea se cierran
  sobre s\'i  mismas o empiezan y terminan en el infinito. 
\item Dibuje varias l\'ineas representativas de flujo del vector de
  Poynting,  incluyendo alguna que pase por el origen de las coordenadas.
\end{itemize}


\item Una part\'icula con carga $q$ y masa $m$ se mueve en campos
  externos est\'aticos ${\bf E_0(r)}$ y ${\bf B_0(r)}$. Si
  $\varphi_0({\bf r_q})$ es el potencial electrost\'atico en la
  posici\'on de la part\'icula, y las perdidas por radiaci\'on son
  despreciables, la expresi\'on usual para la conservaci\'on de
  energ\'ia de este sistema es 
\begin{displaymath}
\frac{1}{2}mv^2 + q\varphi_0({\bf r_q})=constante.
\end{displaymath}

Muestre que esta ecuaci\'on es consistente con el balance de potencia
del teorema de Poynting solamente si se a\~nada la potencia entregada
por una fuente externa de energ\'ia que mantiene los campos contra el
trabajo hecho en sus fuentes por los campos producidos por la
part\'icula. 

\item
Considere un capacitor de placas paralelas con cargas $\pm Q$, \'area
$A$ y separaci\'on $d$ en presencia de un campo magn\'etico ${\bf
  B_0}$. Encuentre el momento lineal almacenado en los campos para
este sistema. 

\item Encuentre el momento angular almacenado en los campos
  para una esfera ferromagn\'etica de metal que tiene
  radio $R$, carga $Q$ y una magnetizaci\'on uniforme ${\bf M}=M{\bf
    \hat{z}}$. 
  \begin{itemize}
    
  \item Compare este valor con el valor del momento angular mec\'anico
    que es adquirido cuando la magnetizaci\'on se quita (por ejemplo
    calentando la esfera sobre la temperatura de Curie haciendo que el
    ferromagnetismo desaparezca). En este caso el cambio del campo
    magn\'etico va a inducir un campo el\'ectrico. Esta ser\'ia una
    versi\'on m\'as f\'isica de la paradoja de Feynman discutida en
    clase. 
    
  \item Compare el valor anterior con el momento angular adquirido
    cuando la esfera se descarga (por ejemplo conectando la esfera a
    tierra). En este caso el cambio del campo el\'ectrico va a inducir
    un campo magn\'etico. 
  \end{itemize}
  
\end{enumerate}



\end{document}
