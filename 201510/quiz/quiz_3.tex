\documentclass[letterpaper,10pt,onecolumn]{article}
\usepackage[spanish]{babel}
\usepackage[latin1]{inputenc}
\usepackage{amsfonts}
\usepackage{amsthm}
\usepackage{amsmath}
\usepackage{mathrsfs}
\usepackage{empheq}
\usepackage{enumitem}
\usepackage[pdftex]{color,graphicx}
\usepackage{hyperref}
\usepackage{listings}
\usepackage{calligra}
\usepackage{algpseudocode} 
\DeclareMathAlphabet{\mathcalligra}{T1}{calligra}{m}{n}
\DeclareFontShape{T1}{calligra}{m}{n}{<->s*[2.2]callig15}{}
\newcommand{\scripty}[1]{\ensuremath{\mathcalligra{#1}}}
\lstloadlanguages{[5.2]Mathematica}
\setlength{\oddsidemargin}{0cm}
\setlength{\textwidth}{490pt}
\setlength{\topmargin}{-40pt}
\addtolength{\hoffset}{-0.3cm}
\addtolength{\textheight}{4cm}

\begin{document}
\begin{center}

\includegraphics[width=490pt]{header.png}\\[0.5cm]

\textsc{\LARGE Quiz 3 - Electromagnetismo II (FISI-3434) - 2015-10}\\[0.5cm]

\textsc{\Large{Profesor: Jaime Forero}} \\[0.5cm]

\textsc{\Large{Febrero 26, 2015}} \\[0.5cm]



\end{center}

Tienen 20 minutos para responder las siguientes 4 preguntas. Solamente
una opci\'on es v\'alida. No es necesario escribir una
justificaci\'on. Los puntos con respuesta correcta valen $+1.25$.


Las primeras dos preguntas hacen referencia 




\begin{enumerate}

\item  
Considere la superposici\'on de
dos ondas planas electromagn\'eticas ortogonales que se pueden
escribir como la parte real de  ${\bf E}={\bf \hat{x}} E_1
  \exp[i(kz-\omega t)] + {\bf \hat{y}}E_2\exp{[i(kz-\omega t+\pi)]}$ donde
$k$, $\omega$, $E_1$ y $E_2$ son reales.

Si $E_2=E_1$, la punta del vector del campo el\'ectrico va a
  describir una trayectoria que vista a lo largo del eje $z$ desde el
  lugar positivo de $z$ hacia el origen se ve como:
\begin{itemize}
\item[a)] Una l\'inea a $45^{\circ}$ del eje $x$ positivo.
\item[b)] Una l\'inea a $135^{\circ}$ del eje $x$ positivo.
\item[c)] Un c\'irculo en direcci\'on horaria.
\item[d)] Un c\'irculo en direcci\'on antihoraria. 
\item[e)] Un camino aleatorio.

\end{itemize}

\item
Las ecuaciones de Maxwell se pueden escribir en la forma que aparece
m\'as abajo. Si existiera la carga magn\'etica y se conservara
�Cu\'ales ecuaciones deber\'ian cambiarse ?

\begin{equation}
\nabla\cdot{\bf E} = \rho/\epsilon_0
\end{equation}
\begin{equation}
\nabla\cdot{\bf B} = 0
\end{equation}
\begin{equation}
\nabla\times{\bf E} = -\frac{\partial {\bf B}}{\partial t}
\end{equation}
\begin{equation}
\nabla\times{\bf B} = \mu_0{\bf j} + \mu_0 \epsilon_0 \frac{\partial
  {\bf E}}{\partial t}
\end{equation}

\begin{itemize}
\item[a)] La 1 solamente.
\item[b)] La 2 Solamente.
\item[c)] La 3 solamente.
\item[d)] La 1 y la 4.
\item[e)] La 2 y la 3.
\end{itemize}

\newpage

\item Una corriente $I$ circula por un cable conductor de forma
  cil\'indrica de longitud $L$, radio $R$, longitud
  $L$ que tiene sus extremos a una diferencia de potencial $V$.
  �C\'uanto vale la magnitud del vector de Poynting en la superficie
  del cable?

\begin{itemize}
\item[a)] $VI$ 
\item[b)] $\frac{VI}{\pi R^2}$
\item[c)] $\frac{VI}{2 \pi R}$
\item[d)] $\frac{VI}{2 \pi R L}$
\item[e)] $\frac{VI}{\pi R^2 L}$
\end{itemize}




\item El momento angular almacenado en los campos el\'ectricos  y magn\'eticos
  para una esfera ferromagn\'etica de metal que tiene radio $R$, carga $Q$ y una magnetizaci\'on uniforme ${\bf M}=M{\bf \hat{z}}$ es proporcional a:
\begin{itemize}
\item[a)] $M Q R^2$
\item[b)] $\epsilon_0 M Q R^2$
\item[c)] $\mu_0 M Q R^2$
\item[d)] $\mu_0 M Q R^2 / \epsilon_0$
\item[e)] $\epsilon \mu_0 M Q R^2$
\end{itemize}

\end{enumerate}
\end{document}
