\documentclass[letterpaper,10pt,onecolumn]{article}
\usepackage[spanish]{babel}
\usepackage[latin1]{inputenc}
\usepackage{amsfonts}
\usepackage{amsthm}
\usepackage{amsmath}
\usepackage{mathrsfs}
\usepackage{empheq}
\usepackage{enumitem}
\usepackage[pdftex]{color,graphicx}
\usepackage{hyperref}
\usepackage{listings}
\usepackage{calligra}
\usepackage{algpseudocode} 
\DeclareMathAlphabet{\mathcalligra}{T1}{calligra}{m}{n}
\DeclareFontShape{T1}{calligra}{m}{n}{<->s*[2.2]callig15}{}
\newcommand{\scripty}[1]{\ensuremath{\mathcalligra{#1}}}
\lstloadlanguages{[5.2]Mathematica}
\setlength{\oddsidemargin}{0cm}
\setlength{\textwidth}{490pt}
\setlength{\textheight}{610pt}
\setlength{\topmargin}{-85pt}
\addtolength{\hoffset}{-0.3cm}
\addtolength{\textheight}{4cm}

\begin{document}
\thispagestyle{empty}
\begin{center}

\includegraphics[width=490pt]{figs/header.png}\\[0.5cm]

\textsc{\LARGE Parcial 3 - Electromagnetismo II (FISI-3434) - 2015-10}\\[0.5cm]

\textsc{\Large{Profesor: Jaime Forero --- Fecha: Mayo 2, 2015}} \\[0.5cm]
\end{center}

\begin{enumerate}

\item (20 puntos) Una part\'icula de carga $q$ se mueve a velocidad
  constante sobre el eje $x$. Calcule los campos magn\'etico y
  el\'ectrico cuando la part\'icula se encuentra en el origen del
  sistema de coordenadas.
  
\item (20 puntos) 
  Para la misma part\'icula del punto anterior
  calcule la potencia total que pasa por
  el plano ubicado en $x=a$ en el mismo momento cuando la part\'icula se
  encuentra en el origen del sistema de coordenadas.

\item (20 puntos) Cuando una onda electromagn\'etica incide
  perpendicularmente sobre un plano conductor en reposo el coeficiente
  de reflexi\'on (el cociente entre la energ\'ia reflejada sobre la
  energ\'ia incidente) es igual a $1$. Utilizando las transformaciones
  de Lorentz para el campo electromagn\'etico, calcule el coeficiente de
  reflexi\'on si el espejo se mueve a una velocidad $v$ en la
  direcci\'on de propagaci\'on de la onda incidente. 

\item (20 puntos) Dos part\'iculas id\'enticas de carga $q$, masa $m$,
  energ\'ia $E_0$ se acercan desde el infinito en una colisi\'on
  frontal a velocidades relativistas. Estime la distancia m\'inima a
  la que se acercan las dos cargas. Haga claras todas sus aproximaciones.

\item
Una densidad del Lagrangiano alternativo para el campo electromagn\'etico es
\begin{displaymath}
{\mathcal L}=-\frac{1}{8\pi}\partial_\alpha A_\beta\partial^\alpha A^{\beta} -\frac{1}{c}J_\alpha A^\alpha
\end{displaymath}
\begin{itemize}
\item (10 puntos) Derive las ecuaciones de movimiento Euler-Lagrange.
\item (10 puntos) Explique bajo cu\'ales suposiciones son las ecuaciones de
  movimiento halladas iguales a las ecuaciones de Maxwell.
\end{itemize}

\item (20 puntos) Demuestre expl\'icitamente que la transformaci\'on
  gauge $A^{\alpha}\rightarrow A^{\alpha}+\partial^\alpha \Lambda$ de
  los potenciales en el Lagrangiano de una part\'icula cargada en un
  campo electromag\'netico generan
  otro Lagrangiano equivalente. La part\'icula tiene carga $q$ y
  velocidad $\vec{v}$. Use el hecho que los Lagrangianos que solamente
  difieren en una derivada total del tiempo de alguna funci\'on de las
  coordenadas y del tiempo, son equivalentes. La funci\'on $\Lambda =
  \Lambda (x^\alpha)$.

\end{enumerate}
\end{document}
