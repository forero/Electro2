\documentclass[letterpaper,10pt,onecolumn]{article}
\usepackage[spanish]{babel}
\usepackage[latin1]{inputenc}
\usepackage{amsfonts}
\usepackage{amsthm}
\usepackage{amsmath}
\usepackage{mathrsfs}
\usepackage{empheq}
\usepackage{enumitem}
\usepackage[pdftex]{color,graphicx}
\usepackage{hyperref}
\usepackage{listings}
\usepackage{calligra}
\usepackage{algpseudocode} 
\DeclareMathAlphabet{\mathcalligra}{T1}{calligra}{m}{n}
\DeclareFontShape{T1}{calligra}{m}{n}{<->s*[2.2]callig15}{}
\newcommand{\scripty}[1]{\ensuremath{\mathcalligra{#1}}}
\lstloadlanguages{[5.2]Mathematica}
\setlength{\oddsidemargin}{0cm}
\setlength{\textwidth}{490pt}
\setlength{\textheight}{610pt}
\setlength{\topmargin}{-85pt}
\addtolength{\hoffset}{-0.3cm}
\addtolength{\textheight}{4cm}

\begin{document}
\thispagestyle{empty}
\begin{center}

\includegraphics[width=490pt]{figs/header.png}\\[0.5cm]

\textsc{\LARGE Parcial 1 - Electromagnetismo II (FISI-3434) - 2015-10}\\[0.5cm]

\textsc{\Large{Profesor: Jaime Forero --- Fecha: Febrero 21, 2015}} \\[0.5cm]
\end{center}

\begin{enumerate}
\item (10 puntos) Un observador $O$ en reposo se encuentra entre dos
  fuentes de luz ubicadas en $x=0$ y $x=10$m y observa que las dos fuentes se
  encienden simult\'aneamente. Un segundo observador $O^\prime$, que
  se mueve a una velocidad constante paralela al eje $x$, observa que
  una fuente de luz se enciende $13$ns antes que la otra. �Cu\'al
  podr\'ia ser la velocidad relativa de $O^{\prime}$ respecto a $O$?
  Exprese el resultado en unidades de la velocidad de la luz $c$.

\item (10 puntos) Dos naves espaciales que vienen de direcciones opuestas se
  acercan a la Tierra con rapideces iguales medidas por una
  observadora en la Tierra. Una varilla en una de las naves espaciales
  mide un metro de longitud (medida en la misma nave), pero mide $80$cm
  por una ocupante de la otra nave. �Cu\'al es la velocidad de cada
  nave medida por un observador en la Tierra?


\item (20 puntos) Considere una c\'amara fotogr\'afica situada sobre
  el eje $y$ positivo de un sistema de referencia inercial y una
  varilla homog\'enea de longitud propia $l_0$ que se mueve con
  velocidad $v$ a lo largo del eleje $x$ positivo. Si la
  c\'amara, situada a una distancia $D$ del origen de coordenadas,
  toma una foto de la varilla de tal manera que el centro de la varilla
  aparezca en el origin de coordenadas �Cu�l es la longitud aparente
  de la varilla seg�na la foto y cu�l es su relaci�n con la longitud
  f�sica? 

\item (20 puntos) Considere el efecto Compton donde un
  fot\'on de longitud de onda $\lambda$ es dispersado por un
  electr\'on en reposo. Sea $\theta$ el \'angulo que forma el fot\'on
  dispersado con la direcci\'on inicial de propagaci\'on y $\phi$ el
  \'angulo de dispersi\'on del electr\'on.  Encuentre $\phi$ en
  funci\'on de: el \'angulo de dispersi\'on del fot\'on $\theta$, la
  masa en reposo del electr\'on $m_{e0}$, la velocidad de la luz $c$ y
  la energ\'ia del fot\'on incidente  $E_{\gamma}$. 


\item (20 puntos) Un prot\'on con $\gamma=1/\sqrt{1-(v^2/c^2)}$
  colisiona el\'asticamente con otro prot\'on en reposo. Si los dos
  protones salen con energ\'ias iguales �Cu\'al es el \'angulo
  $\theta$ entre ellos? 

\item (20 puntos) Una onda plana de frecuencia $\omega$
incide perpendicularmente sobre un espejo que se mueve con una
velocidad $v$ en la direcci\'on de propagaci\'on de la onda. �Cu\'anto vale la
frecuencia de la onda reflejada medida en el sistema de laboratorio?  

\item El efecto Sagnac es un fen\'omeno de interferencia
  que ocurre debido a la rotaci\'on. Si consideramos una gu\'ia de
  luz en forma de anillo y radio $R$ con velocidad angular $\omega$ y
  dos haces de luz que son inyectados sobre el mismo lugar de la gu\'ia con
  direcciones opuestas de propagaci\'on, se puede calcular que el
  desfase entre los dos haces es $\Delta\phi = 2\pi c\Delta t/\lambda$
  donde $\lambda$ es la longitud de onda de la luz y $\Delta
  t=4A\omega/c^2$, con $A=\pi R^2$ el \'area del anillo. Este desfase
  ha sido confirmado experimentalmente a trav\'es de experimentos de
  interferencia.
  \begin{itemize}
    \item (10 puntos) Deduzca con un razonamiento cl\'asico el valor de $\Delta
      t$.
    \item (10 puntos) Explique c\'omo podr\'ia interpretarse este
      efecto desde la relatividad especial. 
  \end{itemize}

 
\end{enumerate}
\end{document}
