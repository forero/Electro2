\documentclass[letterpaper,10pt,onecolumn]{article}
\usepackage[spanish]{babel}
\usepackage[latin1]{inputenc}
\usepackage{amsfonts}
\usepackage{amsthm}
\usepackage{amsmath}
\usepackage{mathrsfs}
\usepackage{empheq}
\usepackage{enumitem}
\usepackage[pdftex]{color,graphicx}
\usepackage{hyperref}
\usepackage{listings}
\usepackage{calligra}
\usepackage{algpseudocode} 
\DeclareMathAlphabet{\mathcalligra}{T1}{calligra}{m}{n}
\DeclareFontShape{T1}{calligra}{m}{n}{<->s*[2.2]callig15}{}
\newcommand{\scripty}[1]{\ensuremath{\mathcalligra{#1}}}
\lstloadlanguages{[5.2]Mathematica}
\setlength{\oddsidemargin}{0cm}
\setlength{\textwidth}{490pt}
\setlength{\textheight}{610pt}
\setlength{\topmargin}{-85pt}
\addtolength{\hoffset}{-0.3cm}
\addtolength{\textheight}{4cm}

\begin{document}
\thispagestyle{empty}
\begin{center}

\includegraphics[width=490pt]{figs/header.png}\\[0.5cm]

\textsc{\LARGE Final - Electromagnetismo II (FISI-3434) - 2015-10}\\[0.5cm]

\textsc{\Large{Profesor: Jaime Forero --- Fecha: Mayo 14, 2015}} \\[0.5cm]
\end{center}

\begin{enumerate}

\item (20 puntos)
 Demuestre que es imposible tener un proceso donde un  prot\'on en
 reposo emite un fot\'on y retrocede.

\item (20 puntos) El campo el\'ectrico de una onda electromag\'etica
  en el vac\'io est\'a descrito por ${\bf E}={\bf \hat{x}} E_1 \cos(\omega t)
  \cos(kz) + {\bf \hat{y}} E_2 \sin(\omega t)
  \sin(kz)$. Calcule el vector de Poynting promediado en el
  tiempo.



\item (20 puntos) 
Luz monocrom\'atica de frecuencia $\omega$ emitida dentro de un acuario pasa por
el agua ($n=n_1$) atraviesa un vidrio plano ($n=n_2$)  de grosor $d$ y
sale al aire ($n=n_3$).  Asumiendo que la luz llega de manera normal al
vidrio encuentre bajo que condici\'ones el coeficiente de
transmisi\'on es m\'aximo (no es necesario calcular el coeficiente). 

\item
Considere la media antena que se muestra en la Figura. 
\begin{center}
\includegraphics[width=300pt]{figs/antena.png}
\end{center}
La corriente se
distribuye por la l\'inea punteada y tienen la siguiente dependencia
temporal y espacial
\begin{displaymath}
I=I_0 \cos(2\pi z/\lambda)\cos(\omega t).
\end{displaymath} 
\begin{itemize}
\item (10 puntos) Encuentre el vector potencial en la zona de radiaci\'on.
\item (10 puntos) Muestre que el promedio temporal de la potencia emitida por
  unidad de \'angulo s\'olido es 
\begin{displaymath}
\frac{dP}{d\Omega} \propto
I_0^2\frac{\cos^{2}((\pi/2)\cos\theta)}{\sin^2{\theta}} 
\end{displaymath}
\end{itemize}

Recuerde que:
\begin{displaymath}
\frac{dP}{d\Omega} \propto
\mathrm{Re}[r^2 {\bf n}\cdot ({\bf E}\times {\bf B}^{*})]
\end{displaymath}



\item Una part\'icula ultrarelativista de masa $m$ y carga $q$ se
  mueve en un plano perpendicular a un campo magn\'etico de magnitud
  $B$ emitiendo radiaci\'on sincrotr\'on.   

\begin{itemize}
\item (15 puntos) Muestre que la energ\'ia de la part\'icula decrece
  en el tiempo de acuerdo a  
\begin{displaymath}
\gamma = \gamma_0(1+A\gamma_0 t)^{-1}, 
\end{displaymath}
donde $\gamma_0$ es el valor inicial de $\gamma$ y $A$ es una
constante que  debe encontrar.
\item (5 puntos) En clase hab\'iamos visto que el factor $\gamma$ es
constante para el movimiento de una part\'icula dentro de un campo
magn\'etico. Explique c\'omo se reconcilia este hecho con el resultado
que acaba de encontrar donde $\gamma$ cambia. 
\end{itemize}

\item (20 puntos) Un positr\'on no relativista de carga $e$ y velocidad
  $v_1$ llega en una colisi\'on frontal sobre un n\'ucleo fijo de
  carga $Ze$. El positr\'on, que viene desde el infinito, se
  desacelera hasta que llega al reposo y luego se vuelve a acelerar
  hasta que llega a una velocidad terminal $v_2$ en el
  infinito. Tomando en cuenta las p\'erdidas por radiaci\'on (pero
  asumiendo que son peque\~nas), encuentre $v_2$ como una funci\'on de
  $v_1$ y Z.



\end{enumerate}
\end{document}
