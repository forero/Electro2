\documentclass[letterpaper,10pt,onecolumn]{article}
\usepackage[spanish]{babel}
\usepackage[latin1]{inputenc}
\usepackage{amsfonts}
\usepackage{amsthm}
\usepackage{amsmath}
\usepackage{mathrsfs}
\usepackage{empheq}
\usepackage{enumitem}
\usepackage[pdftex]{color,graphicx}
\usepackage{hyperref}
\usepackage{listings}
\usepackage{calligra}
\usepackage{algpseudocode} 
\DeclareMathAlphabet{\mathcalligra}{T1}{calligra}{m}{n}
\DeclareFontShape{T1}{calligra}{m}{n}{<->s*[2.2]callig15}{}
\newcommand{\scripty}[1]{\ensuremath{\mathcalligra{#1}}}
\lstloadlanguages{[5.2]Mathematica}
\setlength{\oddsidemargin}{0cm}
\setlength{\textwidth}{490pt}
\setlength{\topmargin}{-40pt}
\addtolength{\hoffset}{-0.3cm}
\addtolength{\textheight}{4cm}

\begin{document}
\begin{center}

\includegraphics[width=490pt]{header.png}\\[0.5cm]

\textsc{\LARGE Electromagnetismo 2}\\[0.1cm]

\large Jaime E. Forero Romero\\[0.5cm]

\end{center}

\large \noindent\textsc{Nombre del curso:}  Electromagnetismo 2%Aqui
                                %nombre del curso 
  
\noindent\textsc{C\'odigo del curso:} FISI 3434 %Aqui el codigo del
                                %curso 

\noindent\textsc{Unidad acad\'emica:} Departamento de F\'isica

\noindent\textsc{Periodo acad\'emico:} 201510 %Aqui el periodo,
                                %p.ej. 201510 

\noindent\textsc{Horario:} Ma y Ju, 8:30 a 9:50%Aqui el horario,
                                %p.ej. Ma y Ju, 10:00 a 11:20 

\noindent\rule{\textwidth}{1pt}\\[-0.3cm]

\normalsize \noindent\textsc{Nombre profesor(a) principal:} Jaime
E. Forero Romero%Aqui nombre del profesor principal 

\noindent\textsc{Correo electr\'onico:}
\href{mailto:je.forero@uniandes.edu.co}{\nolinkurl{je.forero@uniandes.edu.co}}
%Cambie address por su direccion de correo uniandes 

\noindent\textsc{Horario y lugar de atenci\'on:} Ma y Ju 10:00 a
11:00 AM, Oficina Ip208 %Aqui su horario y lugar de atencion, p.ej. Vi,
                     %15:00 a 17:00, Oficina Ip102  
%\\[-0.1cm]

%\noindent\textsc{Nombre profesor(a) complementario(a):} %Aqui nombre
                                %del profesor complementario si aplica 

%\noindent\textsc{Correo electr\'onico:}
%\href{mailto:address@uniandes.edu.co}{\nolinkurl{address@uniandes.edu.co}}
%Cambie address por direccion de correo uniandes del profesor
%complementario 

%\noindent\textsc{Horario y lugar de atenci\'on:} %Aqui horario y
%lugar de atencion del profesor complementario, p.ej. Vi, 15:00 a
%17:00, Oficina Ip102 
%\\[-0.1cm]
%Repetir esto en caso de varios profesores complementarios

%\noindent\textsc{Nombre monitor(a):} %Aqui nombre del monitor si aplica

%\noindent\textsc{Correo electr\'onico:}
%\href{mailto:address@uniandes.edu.co}{\nolinkurl{address@uniandes.edu.co}}
%%Cambie address por direccion de correo uniandes del monitor 

%\noindent\textsc{Horario y lugar de atenci\'on:} %Aqui horario y
%lugar de atencion del monitor, p.ej. Vi, 15:00 a 17:00, Oficina Ip102 

\noindent\rule{\textwidth}{1pt}\\[-0.1cm]

\newcounter{mysection}
\addtocounter{mysection}{1}

\noindent\textbf{\large \Roman{mysection} \quad Introducci\'on}\\[-0.2cm]

%Este espacio es para hacer una introduccion al curso, evidenciando la
%propuesta metodologica. Debe ser clara y precisa. 

\noindent\normalsize El objetivo principal de este curso es presentar fen\'omenos
f\'isicos que necesitan de la aplicaci\'on de los fundamentos vistos
en el curso de Electromagnetismo I para su descripci\'on y
entendimiento. Los principales fen\'omenos que vamos a cubrir son la propagaci\'on
de ondas electromagn\'eticas y la radiaci\'on de cargas en
movimiento. Adicionalmente, en este curso se presentar\'a un
introducc\'on detallada a la relatividad especial para poder abarcar
casos relativistas.

El eje central del curso ser\'a la presentaci\'on de los diferentes
temas acompa\~nados de aplicaciones. Igualmente, procurar\'e
desarrollar una intuici\'on y/o actitud de investigaci\'on pidiendo a
los estudiantes que se enfrenten a problemas que requieren un
esfuerzo por evidenciar el fen\'omeno f\'isico dominante m\'as que
aplicar f\'ormulas y conceptos a problemas de texto.
\\[0.1cm]

\stepcounter{mysection}
\noindent\textbf{\large \Roman{mysection} \quad Objetivos}\\[-0.2cm]

%En este espacio se debe precisar el ente visor del curso y el
%proposito ideal al finalizar el curso. 
\noindent\normalsize Los objetivos principales del curso son:

\begin{itemize}
	\item Analizar las leyes de la electrodin\'amica aplicarlas en
          diferentes situaciones f\'isicas usando m\'etodos
          matem\'aticos apropiados.\\[-0.6cm] 
	\item Estudiar los distintos fen\'omenos ondulatorios del
          electromagnetismo en el vac\'io y en distintos
          materiales.\\[-0.6cm] 
	\item Comprender las ecuaciones de Maxwell desde un punto de
          vista relativista y su formulaci\'on matem\'atica
          correspondiente.\\[-0.2cm] 
\end{itemize}

\stepcounter{mysection}
\noindent\textbf{\large \Roman{mysection} \quad Competencias a
  desarrollar}\\[-0.2cm] 

%En este espacio se describen las habilidades que el estudiante desarrollara en el transcurso del curso.

\noindent\normalsize Al finalizar el curso, se espera que el
estudiante est\'e en capacidad de: 

\begin{itemize}
	\item Aplicar las ecuaciones de Maxwell a diversos fen\'omenos
          ondulatorios en el vac\'io, en medios lineales y en
          conductores.\\[-0.6cm]
	\item Calcular potencias de radiaci\'on electromagn\'etica en
          diferentes situaciones, tanto cl\'asicas como
          relativistas.\\[-0.6cm] 
	\item Aplicar la Teor\'ia Especial de la Relatividad al caso
          electrodin\'amico y algunas de sus aplicaciones.\\[-0.6cm] 
	\item Generar conocimiento a partir del modelamiento te\'orico
          y computacional de los conceptos vistos en clase.\\[-0.2cm] 
\end{itemize}

\stepcounter{mysection}
\noindent\textbf{\large \Roman{mysection} \quad Contenido por
  semanas}\\[-0.2cm] 

%Se expone de forma ordenada toda la tematica a tratar del curso. Debe planearse para 15 semanas.


\noindent\normalsize\textbf{\textsc{Semana 1.}} Cinem\'atica
Relativista. Introducci\'on. Transformaciones de Lorentz.\\[-0.3cm]   

\noindent\textbf{\textsc{Semana 2.}} Cinem\'atica
Relativista. Cuadrivectores. \\[-0.3cm]   

\noindent\textbf{\textsc{Semana 3.}} Din\'amica 
Relativista. Ecuaciones de movimiento y leyes de conservaci\'on
\\[-0.3cm]  

\noindent\textbf{\textsc{Semana 4.}} Din\'amica 
Relativista. Aplicaciones. \\[-0.3cm]  

\noindent\textbf{\textsc{Semana 5.}} Repaso Ecuaciones de
Maxwell. Leyes de Conservaci\'on: carga, energ\'ia y
momentum. {\textbf{Primer Parcial}.}\\[-0.3cm]  

\noindent\textbf{\textsc{Semana 6.}} Ondas
electromagn\'eticas. Funci\'on de onda. Ondas en el vac\'io y
condiciones de frontera (reflexi\'on y transmisi\'on).\\[-0.3cm]  

\noindent\textbf{\textsc{Semana 7.}} Ondas electromagn\'eticas en
la materia (medios lineales). Ecuaciones de Fresnel.\\[-0.3cm] 

\noindent\textbf{\textsc{Semana 8.}} Absorci\'on y
dispersi\'on. Gu\'ia de ondas. {\textbf{Segundo Parcial}.}\\[-0.3cm]

\noindent\textbf{\textsc{Semana 9.}} Formulaci\'on
potencial. Transformaciones Gauge. Potenciales de distribuciones
continuas. \\[-0.3cm] 

\noindent\textbf{\textsc{Semana 10.}} Potenciales de cargas
puntuales. \\[-0.3cm]  

\noindent\textbf{\textsc{Semana 11.}} Radiaci\'on de cargas en
movimiento.  F\'ormula de Larmor. \\[-0.3cm]  

\noindent\textbf{\textsc{Semana 11.}} Radiaci\'on de part\'iculas en
movimiento relativista. Radiaci\'on emitida durante colisiones. \\[-0.3cm]  

\noindent\textbf{\textsc{Semana 13.}} Electrodin\'amica
relativista. {\textbf{Tercer Parcial}.}\\[-0.3cm] 

\noindent\textbf{\textsc{Semana 14.}} Tensor de campo. Potenciales
relativistas. \\[-0.3cm] 

\noindent\textbf{\textsc{Semana 15.}} Aplicaciones y ejemplos de
electrodin\'amica relativista.\\[-0.1cm]  


\stepcounter{mysection}
\noindent\textbf{\large \Roman{mysection} \quad
  Metodolog\'ia}\\[-0.2cm] 

%Se describen las tecnicas y metodos para el desarrollo exitoso del curso.

\noindent\normalsize El curso tendr\'a dos partes importantes. La
primera es el desarrollo de clases magistrales donde se dar\'a
\'enfasis a la aplicaci\'on de conceptos b\'asicos a la resoluci\'on
de problemas. La segunda es la participaci\'on de estudiantes para
resolver ejercicios y problemas. \\[0.1cm]

\stepcounter{mysection}
\noindent\textbf{\large \Roman{mysection} \quad Criterios de
  evaluaci\'on}\\[-0.2cm] 

En el curso se har\'an ocho quizes, tres parciales y un examen
final. Tambi\'en se dar\'an talleres con ejercicios y problemas para
que los estudiantes los trabajen por fuera del horario de
clase. Los quizes, parciales y ex\'amenes reciben
calificaci\'on. 

Adicionalmente, habr\'a un espacio para la participaci\'on de
los estudiantes en la forma de resoluci\'on de problemas en el
tablero. Esta participaci\'on tambi\'en recibe calificaci\'on.

Los porcentajes de cada evaluaci\'on son los siguientes.
\begin{itemize}
\item Primer parcial: 15\%
\item Segundo parcial: 15\%
\item Tercer parcial: 15\%
\item Promedio de quizes (se quitan la mejor y la peor nota): 15\%
\item Promedio participaci\'on en clase (se quitan la mejor y la peor
  nota): 20\% 
\item Examen final: 20\%
\end{itemize}



%Tener en cuenta los siguientes aspectos:
%	\item Porcentajes de cada evaluacion
%	\item Fechas importantes
%	\item Parametros de calificacion
%	\item Calificacion de asistencia y/o participacion en clase
%	\item Reclamos
%	\item Politica de aproximacion de notas



\stepcounter{mysection}
\noindent\textbf{\large \Roman{mysection} \quad
  Bibliograf\'ia}\\[-0.2cm] 

%Indicar los libros y la documentacion guia.

\noindent\normalsize Bibliograf\'ia principal:

\begin{itemize}
	\item D.J. Griffiths. \textit{Introduction to
          Electrodynamics}, 1999. (Biblioteca General - 537.6 G633
          1999) 
        \item L.D. Landau, E.M. Lifshitz. \textit{The Classical
          Theory of Fields. Vol. 2}, (4a ed.), 1975.
        \item J.M. Tejeiro \textit{Sobre la teor\'ia especial de la
          relatividad}, 2004, Notas de clase, versi\'on en l\'inea:
          \url{https://gnfisica.files.wordpress.com/2010/08/sobre_la_teoria_relatividadtejeiro.pdf} 
	\item R.P. Feynman, \textit{The Feynman Lectures on Physics},
          2006. Disponible online en
          \href{http://www.feynmanlectures.caltech.edu}{\nolinkurl{http://www.feynmanlectures.caltech.edu}}. (Biblioteca
          General - 530.0711 F295 2006)\\[-0.6cm] 
	\item J.D. Jackson. \textit{Classical Electrodynamics},
          1999. (Biblioteca General - 537.6 J114 1999)\\[-0.6cm] 
\end{itemize}

\noindent\normalsize Bibliograf\'ia complementaria:

\begin{itemize}
	\item E.M. Purcell. \textit{Electricity and Magnetism},
          1985. (Biblioteca General - 537.1 P971 1985)\\[-0.6cm] 
	\item J.R. Reitz, F.J. Milford y
          R.W. Christy. \textit{Foundations of Electromagnetic
            Theory}, 1993. (Biblioteca General - 530.141 R237
          1993)\\[-0.6cm] 
	\item P. Lorrain, D.R. Corson. \textit{Electromagnetism,
          Principles and Applications}, 1979. (Biblioteca General -
          537. L561 1979)\\[-0.6cm] 
	\item J. Vanderlinde. \textit{Classical Electromagnetic Theory}, 2005. Disponible online (dentro del campus) en Springerlink:\\
	\href{http://link.springer.com/book/10.1007/1-4020-2700-1}{\nolinkurl{http://link.springer.com/book/10.1007/1-4020-2700-1}} 
\end{itemize}

\end{document}
