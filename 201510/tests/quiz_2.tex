\documentclass[letterpaper,10pt,onecolumn]{article}
\usepackage[spanish]{babel}
\usepackage[latin1]{inputenc}
\usepackage{amsfonts}
\usepackage{amsthm}
\usepackage{amsmath}
\usepackage{mathrsfs}
\usepackage{empheq}
\usepackage{enumitem}
\usepackage[pdftex]{color,graphicx}
\usepackage{hyperref}
\usepackage{listings}
\usepackage{calligra}
\usepackage{algpseudocode} 
\DeclareMathAlphabet{\mathcalligra}{T1}{calligra}{m}{n}
\DeclareFontShape{T1}{calligra}{m}{n}{<->s*[2.2]callig15}{}
\newcommand{\scripty}[1]{\ensuremath{\mathcalligra{#1}}}
\lstloadlanguages{[5.2]Mathematica}
\setlength{\oddsidemargin}{0cm}
\setlength{\textwidth}{490pt}
\setlength{\topmargin}{-40pt}
\addtolength{\hoffset}{-0.3cm}
\addtolength{\textheight}{4cm}

\begin{document}
\begin{center}

\includegraphics[width=490pt]{header.png}\\[0.5cm]

\textsc{\LARGE Quiz 2 - Electromagnetismo II (FISI-3434) - 2015-10}\\[0.5cm]

\textsc{\Large{Profesor: Jaime Forero}} \\[0.5cm]

\textsc{\Large{Febrero 12, 2015}} \\[0.5cm]



\end{center}

Tienen 20 minutos para responder las siguientes 4 preguntas. Solamente
una opci\'on es v\'alida. No es necesario escribir una
justificaci\'on. Los puntos con respuesta correcta valen $+1.25$ y con
respuesta incorrecta valen $-0.625$. 


\begin{enumerate}

\item Un haz de muones se mueve por el laboratorio con velocidad
  $4/5c$. La vida media del mu\'on en su sistema de reposo es
  $\tau=2.2\times 10^{-6}$s. La distancia media recorrida por los
  muones en el marco de referencia del laboratorio es:
\begin{itemize}
\item[a)] 530 m
\item[b)] 660 m
\item[c)] 880 m
\item[d)] 1100 m
\item[e)] 1500 m
\end{itemize}

\item Una galaxia distante se observa con una l\'inea de emisi\'on de
  hidr\'ogeno-$\beta$ corrida a una longitud de onda de $580$nm con
  respecto a la longitud de onda medida en el laboratorio de
  $434$nm. �Cu\'al es aproximadamente la velocidad de
  recesi\'on de esta galaxia distante?
\begin{itemize}
\item[a)] $0.28c$
\item[b)] $0.53c$
\item[c)] $0.56c$
\item[d)] $0.75c$
\item[e)] $0.86c$
\end{itemize} 

\item Una part\'icula de masa $M$ decae del reposo en dos
  part\'iculas. Una de ellas tiene masa $m$ y la otra no tiene
  masa. �Cu\'al es el momentum de la part\'icula sin masa?
\begin{itemize}
\item[a)] $\frac{(M^2 -m^2)c}{4M}$
\item[b)] $\frac{(M^2 -m^2)c}{2M}$
\item[c)] $\frac{(M^2 -m^2)c}{M}$
\item[d)] $\frac{2(M^2 -m^2)c}{M}$
\item[d)] $\frac{4(M^2 -m^2)c}{M}$
\end{itemize}


\item Dos naves espaciales que vienen de direcciones opuestas se
  acercan a la Tierra con rapideces iguales medidas por una
  observadora en la Tierra. Una varilla en una de las naves espaciales
  mide un metro de longitud (medida en la misma nave), pero mide $60$cm
  por una ocupante de la otra nave. �Cu\'al es la velocidad de cada
  nave medida por un observador en la Tierra?

\begin{itemize}
\item[a)] $0.4c$
\item[b)] $0.5c$
\item[c)] $0.6c$
\item[d)] $0.7c$
\item[e)] $0.8c$
\end{itemize} 
\end{enumerate}

\end{document}
