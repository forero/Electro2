\documentclass[letterpaper,10pt,onecolumn]{article}
\usepackage[spanish]{babel}
\usepackage[latin1]{inputenc}
\usepackage{amsfonts}
\usepackage{amsthm}
\usepackage{amsmath}
\usepackage{mathrsfs}
\usepackage{empheq}
\usepackage{enumitem}
\usepackage[pdftex]{color,graphicx}
\usepackage{hyperref}
\usepackage{listings}
\usepackage{calligra}
\usepackage{algpseudocode} 
\DeclareMathAlphabet{\mathcalligra}{T1}{calligra}{m}{n}
\DeclareFontShape{T1}{calligra}{m}{n}{<->s*[2.2]callig15}{}
\newcommand{\scripty}[1]{\ensuremath{\mathcalligra{#1}}}
\lstloadlanguages{[5.2]Mathematica}
\setlength{\oddsidemargin}{0cm}
\setlength{\textwidth}{490pt}
\setlength{\topmargin}{-40pt}
\addtolength{\hoffset}{-0.3cm}
\addtolength{\textheight}{4cm}

\begin{document}
\begin{center}

\includegraphics[width=490pt]{header.png}\\[0.5cm]

\textsc{\LARGE Quiz 1 - Electromagnetismo II (FISI-3434) - 2015-10}\\[0.5cm]

\textsc{\Large{Profesor: Jaime Forero}} \\[0.5cm]

\textsc{\Large{Febrero 5, 2015}} \\[0.5cm]



\end{center}

Tienen 12 minutos para responder las siguientes 4 preguntas. Solamente
una opci\'on es v\'alida. No es necesario escribir una
justificaci\'on.




\begin{enumerate}
\item Un observador $O$ en reposo se encuentra entre dos fuentes de luz
  ubicadas en $x=0$ y $x=10$m y observa que las dos fuentes se
  encienden simult\'aneamente. Un segundo observador
  $O^\prime$, que se mueve a una velocidad constante paralela al eje
  $x$, observa que una fuente de luz se enciende $13$ns antes que la
  otra. Cual de las siguientes opciones corresponde a la velocidad
  relativa de $O^{\prime}$ respecto a $O$? 

\begin{itemize}
\item[a)] 0.13c
\item[b)] 0.15c
\item[c)] 0.36c
\item[d)] 0.53c
\item[e)] 0.62c
\end{itemize}

\item Una varilla  de un metro (en reposo) se mueve con una velocidad
  de $0.8c$ al lado de un observador. En el sistema de referencia del
  observador �cu�nto se demora la varilla en pasar al observador?
\begin{itemize}
  \item[a)] 1.6 ns
  \item[b)] 2.5 ns
  \item[c)] 4.2 ns
  \item[d)] 6.9 ns
  \item[e)] 8.3 ns
\end{itemize}

\item En un sistema inercial de referencia $\Sigma$ dos eventos
  suceden sobre el eje $x$ separados en tiempo por $\Delta t$ y
  en espacio por $\Delta x$. En otro sistema inercial de referencia
  $\Sigma^{\prime}$, que se mueve en la direcci\'on $x$ relativo a
  $\Sigma$, los dos eventos pueden suceder al mismo tiempo, bajo cu\'al
  condici\'on: 
\begin{itemize}
\item[a)] Para cualquier valor de $\Delta x$ y $\Delta t$ es posible
  que pasen al mismo tiempo.
\item[b)] Solamente si $|\Delta x/\Delta t|<c$.
\item[c)] Solamente si $|\Delta x/\Delta t|>c$.
\item[d)] Solamente si $|\Delta x/\Delta t|=c$.
\item[e)] Es imposible que ocurran al mismo tiempo.
\end{itemize}

\item Un pi\'on cargado decae en $10^{-8}$ segundos en un sistema de
  referencia en reposo. Si este pi\'on logra viajar $30$ metros en el
  sistema de referencia del laboratorio, la velocidad del pi\'on debe
  ser cercana a:
\begin{itemize}
\item[a)] $0.43\times 10^{8}$ m/s
\item[b)] $2.84\times 10^{8}$ m/s
\item[c)] $2.90\times 10^{8}$ m/s
\item[d)] $2.98\times 10^{8}$ m/s
\item[e)] $3.00\times 10^{8}$ m/s
\end{itemize}

\end{enumerate}



\end{document}
