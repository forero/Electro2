\documentclass[letterpaper,10pt,onecolumn]{article}
\usepackage[spanish]{babel}
\usepackage[latin1]{inputenc}
\usepackage{amsfonts}
\usepackage{amsthm}
\usepackage{amsmath}
\usepackage{mathrsfs}
\usepackage{empheq}
\usepackage{enumitem}
\usepackage[pdftex]{color,graphicx}
\usepackage{hyperref}
\usepackage{listings}
\usepackage{calligra}
\usepackage{algpseudocode} 
\DeclareMathAlphabet{\mathcalligra}{T1}{calligra}{m}{n}
\DeclareFontShape{T1}{calligra}{m}{n}{<->s*[2.2]callig15}{}
\newcommand{\scripty}[1]{\ensuremath{\mathcalligra{#1}}}
\lstloadlanguages{[5.2]Mathematica}
\setlength{\oddsidemargin}{0cm}
\setlength{\textwidth}{490pt}
\setlength{\topmargin}{-40pt}
\addtolength{\hoffset}{-0.3cm}
\addtolength{\textheight}{4cm}

\begin{document}
\begin{center}

\includegraphics[width=490pt]{header.png}\\[0.5cm]

\textsc{\LARGE Electromagnetismo 2}\\[0.1cm]

\large Jaime E. Forero Romero\\[0.5cm]

\end{center}

\large \noindent\textsc{Nombre del curso:}  Electromagnetismo 2%Aqui nombre del curso
 
\noindent\textsc{C\'odigo del curso:} FISI 3434 %Aqui el codigo del curso

\noindent\textsc{Unidad acad\'emica:} Departamento de F\'isica

\noindent\textsc{Periodo acad\'emico:} 201510 %Aqui el periodo, p.ej. 201510

\noindent\textsc{Horario:} Ma y Ju, 8:30 a 9:50%Aqui el horario, p.ej. Ma y Ju, 10:00 a 11:20

\noindent\rule{\textwidth}{1pt}\\[-0.3cm]

\normalsize \noindent\textsc{Nombre profesor(a) principal:} Jaime E. Forero Romero%Aqui nombre del profesor principal

\noindent\textsc{Correo electr\'onico:} \href{mailto:je.forero@uniandes.edu.co}{\nolinkurl{je.forero@uniandes.edu.co}} %Cambie address por su direccion de correo uniandes

\noindent\textsc{Horario y lugar de atenci\'on:} Ma y Ju 17:00 a
18:00, Oficina Ip208 %Aqui su horario y lugar de atencion, p.ej. Vi,
                     %15:00 a 17:00, Oficina Ip102  
%\\[-0.1cm]

%\noindent\textsc{Nombre profesor(a) complementario(a):} %Aqui nombre
                                %del profesor complementario si aplica 

%\noindent\textsc{Correo electr\'onico:}
%\href{mailto:address@uniandes.edu.co}{\nolinkurl{address@uniandes.edu.co}}
%Cambie address por direccion de correo uniandes del profesor
%complementario 

%\noindent\textsc{Horario y lugar de atenci\'on:} %Aqui horario y lugar de atencion del profesor complementario, p.ej. Vi, 15:00 a 17:00, Oficina Ip102
%\\[-0.1cm]
%Repetir esto en caso de varios profesores complementarios

%\noindent\textsc{Nombre monitor(a):} %Aqui nombre del monitor si aplica

%\noindent\textsc{Correo electr\'onico:} \href{mailto:address@uniandes.edu.co}{\nolinkurl{address@uniandes.edu.co}} %Cambie address por direccion de correo uniandes del monitor

%\noindent\textsc{Horario y lugar de atenci\'on:} %Aqui horario y lugar de atencion del monitor, p.ej. Vi, 15:00 a 17:00, Oficina Ip102

\noindent\rule{\textwidth}{1pt}\\[-0.1cm]

\newcounter{mysection}
\addtocounter{mysection}{1}

\noindent\textbf{\large \Roman{mysection} \quad Introducci\'on}\\[-0.2cm]

%Este espacio es para hacer una introduccion al curso, evidenciando la propuesta metodologica. Debe ser clara y precisa.

\noindent\normalsize Aqu\'i texto.\\[0.1cm]

\stepcounter{mysection}
\noindent\textbf{\large \Roman{mysection} \quad Objetivos}\\[-0.2cm]

%En este espacio se debe precisar el ente visor del curso y el proposito ideal al finalizar el curso.

\noindent\normalsize Los objetivos principales del curso son:

\begin{itemize}
	\item .\\[-0.6cm]
	\item .\\[-0.2cm]
\end{itemize}

\stepcounter{mysection}
\noindent\textbf{\large \Roman{mysection} \quad Competencias a desarrollar}\\[-0.2cm]

%En este espacio se describen las habilidades que el estudiante desarrollara en el transcurso del curso.

\noindent\normalsize Al finalizar el curso, se espera que el estudiante est\'e en capacidad de:

\begin{itemize}
	\item .\\[-0.6cm]
	\item .\\[-0.6cm]
	\item .\\[-0.2cm]
\end{itemize}

\stepcounter{mysection}
\noindent\textbf{\large \Roman{mysection} \quad Contenido por semanas}\\[-0.2cm]

%Se expone de forma ordenada toda la tematica a tratar del curso. Debe planearse para 15 semanas.

\noindent\normalsize \textbf{\textsc{Semana 1.}} .\\[-0.3cm]

\noindent\textbf{\textsc{Semana 2.}} .\\[-0.3cm]

\noindent\textbf{\textsc{Semana 3.}} .\\[-0.3cm]

\noindent\textbf{\textsc{Semana 4.}} .\\[-0.3cm]

\noindent\textbf{\textsc{Semana 5.}} .\\[-0.3cm]

\noindent\textbf{\textsc{Semana 6.}} .\\[-0.3cm]

\noindent\textbf{\textsc{Semana 7.}} .\\[-0.3cm]

\noindent\textbf{\textsc{Semana 8.}} .\\[-0.3cm]

\noindent\textbf{\textsc{Semana 9.}} .\\[-0.3cm]

\noindent\textbf{\textsc{Semana 10.}} .\\[-0.3cm]

\noindent\textbf{\textsc{Semana 11.}} .\\[-0.3cm]

\noindent\textbf{\textsc{Semana 12.}} .\\[-0.3cm]

\noindent\textbf{\textsc{Semana 13.}} .\\[-0.3cm]

\noindent\textbf{\textsc{Semana 14.}} .\\[-0.3cm]

\noindent\textbf{\textsc{Semana 15.}} .\\[0.1cm]

\stepcounter{mysection}
\noindent\textbf{\large \Roman{mysection} \quad Metodolog\'ia}\\[-0.2cm]

%Se describen las tecnicas y metodos para el desarrollo exitoso del curso.

\noindent\normalsize Aqu\'i texto.\\[0.1cm]

\stepcounter{mysection}
\noindent\textbf{\large \Roman{mysection} \quad Criterios de evaluaci\'on}\\[-0.2cm]

%Tener en cuenta los siguientes aspectos:
%	\item Porcentajes de cada evaluacion
%	\item Fechas importantes
%	\item Parametros de calificacion
%	\item Calificacion de asistencia y/o participacion en clase
%	\item Reclamos
%	\item Politica de aproximacion de notas

\noindent\normalsize Aqu\'i texto.\\[0.1cm]

\stepcounter{mysection}
\noindent\textbf{\large \Roman{mysection} \quad Bibliograf\'ia}\\[-0.2cm]

%Indicar los libros y la documentacion guia.

\noindent\normalsize Bibliograf\'ia principal:

\begin{itemize}
	\item Autores. \textit{Titulo}, Anho. (Biblioteca General - Codigo biblio)
\end{itemize}

\noindent\normalsize Bibliograf\'ia complementaria:

\begin{itemize}
	\item Autores. \textit{Titulo}, A\~no. (Biblioteca General - Codigo biblio)\\[-0.6cm]
	\item Autores. \textit{Titulo}, A\~no. (Biblioteca General - Codigo biblio)\\[-0.6cm]
	\item Autores. \textit{Titulo}, A\~no. (Biblioteca General - Codigo biblio)
\end{itemize}

\end{document}
